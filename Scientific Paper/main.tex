% ==============================================================================
%  ____       _            _   _  __ _        ____
% / ___|  ___(_) ___ _ __ | |_(_)/ _(_) ___  |  _ \ __ _ _ __   ___ _ __
% \___ \ / __| |/ _ \ '_ \| __| | |_| |/ __| | |_) / _` | '_ \ / _ \ '__|
%  ___) | (__| |  __/ | | | |_| |  _| | (__  |  __/ (_| | |_) |  __/ |
% |____/ \___|_|\___|_| |_|\__|_|_| |_|\___| |_|   \__,_| .__/ \___|_|
%                                                       |_|
% ==============================================================================



% ==============================================================================
%  ____       _   _   _                      __  ____            _
% / ___|  ___| |_| |_(_)_ __   __ _ ___     / / |  _ \ __ _  ___| | ____ _  __ _  ___  ___
% \___ \ / _ \ __| __| | '_ \ / _` / __|   / /  | |_) / _` |/ __| |/ / _` |/ _` |/ _ \/ __|
%  ___) |  __/ |_| |_| | | | | (_| \__ \  / /   |  __/ (_| | (__|   < (_| | (_| |  __/\__ \
% |____/ \___|\__|\__|_|_| |_|\__, |___/ /_/    |_|   \__,_|\___|_|\_\__,_|\__, |\___||___/
%                             |___/                                        |___/
% ==============================================================================

% ==============================================================================
% General / Allgemein
% ==============================================================================

\documentclass[
    paper=a4,           % Layout Din A4.
    oneside,            % Ändern auf twoside und % vor den \cleardoubleemptypage Befehlen entfernen für Buchlayout.
    fontsize=12pt,      % Standartschriftgröße.
    headinclude,        % Kopfzeile wird beim Seiten-Layouts mit berücksichtigt.
    headsepline,        % Horizontale Linie unter Kolumnentitel.
    plainheadsepline,   % Horizontale Linie auch beim plain-Style.
    open=any,           % Kapitel können auf geraden und ungeraden Seiten beginnen.
    listof=totoc,       % Abbildungsverzeichnis und Tabellenverzeichnis im Inhaltsverzeichnis.
    bibliography=totoc, % Literaturverzeichnis wird in das Inhaltsverzeichnis eingetragen.
    numbers=noenddot    % Kein Punkt nach der letzter Kapitelzahl.
]{scrbook}

\usepackage{hyperref} % Einstellungen des PDF Dokumentes. Links und Verweise erzeugen.
\hypersetup {
    pdfstartview = {FitV},                      % Legt fest wie das PDF geöffnet wird.
    pdfview = {FitH},                           % Legt fest wie das PDF geöffnet wird.
    pdffitwindow = {true}                       % PDF ins Fenster einpassen.
    breaklinks = {true},                        % Erlaubt Zeilenumbrüche in Links.
    colorlinks = {true},                        % Erlaubt farbige Links.
    linkcolor = {link_color},                   % Farbe für Inhaltsverzeichnis und Seitenzahlen.
    citecolor = {link_color},                   % Farbe für Zitate.
    urlcolor = {link_color},                    % Farbe für Hyperlinks.
    pdftitle = {TitelDerArbeitHierEintragen},   % Titel der Arbeit.
    pdfsubject = {ThemaDerArbeitHierEintragen}, % Thema der Arbeit.
    pdfauthor = {NameDesAutorsHierEintragen}    % Autor/in der Arbeit.
}

% ==============================================================================
% Language / Sprache
% ==============================================================================

\usepackage[latin1]{inputenc} % Input-Encodung: latin1 für Unix.
\usepackage[T1]{fontenc}      % T1-kodierte Schriften, korrekte Trennmuster für Worte mit Umlaute.
\usepackage[ngerman]{babel}   % Neue Rechtschreibung.

% ==============================================================================
% Symbols / Symbole
% ==============================================================================

\usepackage{amsmath}  % Erweiterung für den Mathe-Satz.
\usepackage{amsfonts} % Zusätzliche mathematische Symbole.
\usepackage{textcomp} % Stellt unter anderem das Symbol Copyright zur Verfügung.
\usepackage{wasysym}  % Verschiedene Symbole.

% ==============================================================================
% Page formatting / Seitenformatierung
% ==============================================================================

\usepackage{geometry}         % Ermöglicht Randeinstellungen individuell zu ändern.
\usepackage{scrlayer-scrpage} % Kopf- und Fußzeilen-Layout.
\usepackage{lscape}           % Ermöglicht Seiten im Querformat.

\usepackage{chngcntr}              % Counter für durchgängige Nummerierung.
\counterwithout{figure}{chapter}   % Bilder Zählwert in arabische Ziffern.
\counterwithout{table}{chapter}    % Tabellen Zählwert in arabische Ziffern.
\counterwithout{equation}{chapter} % Formal Zählwert in arabische Ziffern.

\usepackage[titles]{tocloft}                                      % Bewirkt das Titel der ToC-, LoF- und LoT-Listen mit Standard-LATEX-Methoden gesetzt werden.
\addtokomafont{sectioning}{\normalcolor\bfseries\fontfamily{phv}} % Kapitelüberschrift in Fett.
\renewcommand{\thefigure}{\bfseries\arabic{figure}}               % Bilder bezifferung erfolgt in Arabische Ziffern.
\renewcommand{\thetable}{\bfseries\arabic{table}}                 % Tabellen bezifferung erfolgt in Arabische Ziffern.

% ==============================================================================
% Header and Footer / Kopf- und Fußzeile
% ==============================================================================

\pagestyle{scrheadings}                        % Stil der Kopf- und Fußzeilen.
\renewcommand{\headfont}{\normalfont\sffamily} % Kolumnentitel serifenlos.
\renewcommand{\pnumfont}{\normalfont\sffamily} % Seitennummern serifenlos.

% Header / Kopfzeile
% ==================================================
\ihead[]{\headmark}          % Innen (Kolumnentitel immer oben).
\chead[]{}                   % Mitte.
\ohead[\pagemark]{\pagemark} % Außen (Seitennummern immer oben).

% Footer / Fußzeile
% ==================================================
\ifoot[]{} % Innen.
\cfoot[]{} % Mitte.
\ofoot[]{} % Außen.

% ==============================================================================
% Text formatting / Textformatierung
% ==============================================================================

\usepackage{float}                   % Erlaubt die genaue Plazierung von Gleitobjekten mit H.
\usepackage{scrhack}                 % Unterdrückt eine Warnung bei Verwendung von float und/oder listings.

\usepackage{parskip}                 % Erste Zeile nicht einrücken.
\usepackage{setspace}                % Zeilenabstand einstellbar.
\onehalfspacing                      % Zeilenabstand auf eineinhalbzeilig einstellen.
\usepackage{multicol}                % Erlaubt mehrspaltiger Text.
\usepackage{multirow}                % Erlaubt mehrzeiliger Text.

\usepackage{xcolor}                  % Ermöglicht Schriftfarben zu ändern.
\definecolor{link_color}{rgb}{0,0,0} % Textfarbe Links (schwarz).

\usepackage{caption}                 % Ermöglicht Untertitel zu formatieren.
\usepackage{subcaption}              % Ermöglicht UnterUntertitel zu formatieren.
\captionsetup{
    justification={centering},
    singlelinecheck={false},
    textformat={period},
    font={footnotesize},
    labelfont={bf},
    labelsep={colon},
    figurename={Abb.},
    tablename={Tab.}
}                                    % Aussehen Untertitel.

\usepackage{paralist}                % Erweiterung / Modifikation der Listenumgebungen.

% ==============================================================================
% Images / Bilder
% ==============================================================================

\usepackage{graphicx}            % Ermöglicht Bilder einzubinden.
\renewcommand{\figurename}{Abb.} % Umbenennung: Abbildung => Abb.
\usepackage[export]{adjustbox}   % Makros um Inhalt von Boxen anzupassen.
\usepackage{epstopdf}            % Konvertierung von EPS (Encapsulated Postscript) zu PDF vor dem kompilieren.

% ==============================================================================
% Spreadsheets / Tabellen
% ==============================================================================

\usepackage{colortbl}                                     % Möglichkeit Reihen/Spalten/einzelne Zellen farblich gestalten.
\usepackage{tabularx}                                     % Möglichkeit Tabellen eine bestimmte Breite vorzugeben.
\newcolumntype{L}[1]{>{\raggedright\arraybackslash}p{#1}} % Linksbündig mit Breitenangabe.
\newcolumntype{C}[1]{>{\centering\arraybackslash}p{#1}}   % Zentriert mit Breitenangabe.
\newcolumntype{R}[1]{>{\raggedleft\arraybackslash}p{#1}}  % Rechtsbündig mit Breitenangabe.

% ==============================================================================
% Literature and References / Literatur und Referenzen
% ==============================================================================

\usepackage[
  backend=biber,
  style=numeric
]{biblatex}                                           % Biblatex Einstellungen.
\addbibresource{./assets/literature/bibliography.bib} % Bibliography Datei.
\usepackage{csquotes}                                 % Helper Package.

% ==============================================================================
%  ____                                        _
% |  _ \  ___   ___ _   _ _ __ ___   ___ _ __ | |_
% | | | |/ _ \ / __| | | | '_ ` _ \ / _ \ '_ \| __|
% | |_| | (_) | (__| |_| | | | | | |  __/ | | | |_
% |____/ \___/ \___|\__,_|_| |_| |_|\___|_| |_|\__|
% ==============================================================================

\begin{document}

    \pagenumbering{Roman} % Seitennummerierung in römische Ziffern.

    % Cover Page / Deckblatt
    % ==================================================
    \newgeometry{top=1.5cm, bottom=2cm, left=1.8cm, right=1cm} % Ränder Deckblatts.
    % ==============================================================================
%  ____            _    _     _       _   _
% |  _ \  ___  ___| | _| |__ | | __ _| |_| |_
% | | | |/ _ \/ __| |/ / '_ \| |/ _` | __| __|
% | |_| |  __/ (__|   <| |_) | | (_| | |_| |_
% |____/ \___|\___|_|\_\_.__/|_|\__,_|\__|\__|
% ==============================================================================
\thispagestyle{empty}
{
\fontfamily{phv}\selectfont % phv = Helvetica
% ==============================================================================

% ==============================================================================
% Logos
% ==============================================================================

\hspace{-0.5cm}
\begin{figure}
  \begin{subfigure}{0.45\textwidth}
    \centering
    \includegraphics[width=0.8\textwidth]{assets/images/logo.png}
  \end{subfigure}
  \begin{subfigure}{0.45\textwidth}
    \centering
    \includegraphics[width=0.8\textwidth]{assets/images/logo.png}
  \end{subfigure}
\end{figure}

% ==============================================================================
\begin{center}
% ==============================================================================

% ==============================================================================
% Type and title of the thesis
% ==============================================================================

\vspace{3cm}
{\large \textbf{TYPE}}

\vspace{0.5cm}
\parbox{0.7\textwidth}{
  \begin{center}
    \huge \textbf{TITLE}
  \end{center}
}

% ==============================================================================
% Degree
% ==============================================================================

\vspace{3cm}
\parbox{0.7\textwidth}{
  \begin{center}
    zur Erlangung des akademischen Grades\\
    \textbf{Bachelor of Engineering (B. Eng.)}\\
    im Studiengang Werkstofftechnik Glas und Keramik\\
  \end{center}
}

\vspace{1cm}
\parbox{0.7\textwidth}{
  \begin{center}
    an der Hochschule Koblenz\\
    WesterWaldCampus H\"ohr-Grenzhausen\\
    Fachbereich Ingenieurwesen
  \end{center}
}

% ==============================================================================
% Author
% ==============================================================================

\vfill
{\fontsize{12}{12} \selectfont
  vorgelegt von\\
  \textbf{AUTHOR}
}\\
\vspace{1cm}

% ==============================================================================
% Supervisors
% ==============================================================================

\hrule
\vspace{1cm}
{\fontsize{12pt}{12} \selectfont
  \begin{tabular}{lcl}
    Betreuer       &:& Herr Prof. Dr. Max Mustermann\\[0.5ex]
                   &:& Frau Prof. Dr. Petra M\"uller\\[0.5ex]
    Eingereicht am &:& \today
  \end{tabular}
}

% ==============================================================================
\end{center}
}
% ==============================================================================


    % Restore defaults / Standarts wiederherstellen
    % ==================================================
    \restoregeometry         % Standartränder wiederherstellen.
    \setcounter{figure}{0}   % Abbildungszähler resetten.
    \setcounter{table}{0}    % Tabellenzähler resetten.
    \setcounter{equation}{0} % Formelzähler resetten.

    % Blank page / Leerseite
    % ==================================================
    \newpage
    \thispagestyle{empty}
    \quad
    \newpage

    % Declaration / Erklärung
    % ==================================================
    %\cleardoubleemptypage
    % ==============================================================================
%  _____      _    _ _   _
% | ____|_ __| | _| (_)_(_)_ __ _   _ _ __   __ _
% |  _| | '__| |/ / |/ _` | '__| | | | '_ \ / _` |
% | |___| |  |   <| | (_| | |  | |_| | | | | (_| |
% |_____|_|  |_|\_\_|\__,_|_|   \__,_|_| |_|\__, |
%                                           |___/
% ==============================================================================
\chapter*{Erkl\"arung zur Abschlussarbeit}
\label{cha:Erkl\"arung}
% ==============================================================================

\begin{tabular}{lp{1cm}l}
    \hspace{6cm} && \hspace{6cm} \\ \cline{1-1} \cline{3-3}
    Name && Vorname
\end{tabular}

\bigskip

\begin{tabular}{l}
    \hspace{6cm} \\ \cline{1-1}
    Studiengang
\end{tabular}

\bigskip
\bigskip

Ich versichere, dass ich die Arbeit vollst\"andig alleine und ohne unzul\"assige Hilfsmittel erstellt habe.

\bigskip
\bigskip

Zutreffendes bitte ankreuzen:\\
Mit der hochschulinternen Ver\"offentlichung der Arbeit bin ich\\
\Square \hspace{0.2cm} einverstanden. \hspace{1cm} \Square \hspace{0.2cm} n i c h t einverstanden.

\bigskip
\bigskip

Ich bin damit einverstanden, dass\\
(Druckform)
\begin{itemize}
    \item der Titel meiner Arbeit mit meinem Namen und denen der Betreuer im Bibliothekskatalog (OPAC) ver\"offentlicht wird und
    \item die Arbeit zur Einsichtnahme in der Bibliothek oder im Fachbereich ausliegt.
\end{itemize}
(Elektronische Form)
\begin{itemize}
    \item die Arbeit als PDF hochschulintern eingesehen werden kann.
\end{itemize}

\bigskip
\bigskip

Daf\"ur erh\"alt die Hochschule ein einfaches, nicht \"ubertragbares Nutzungsrecht ausschlie{\ss}lich f\"ur den Zweck der Ver\"offentlichung in der Bibliothek. Das Recht der Ver\"offentlichung oder Verwertung durch den Verfasser oder die Verfasserin auf andere Weise, z.B. \"uber einen Verlag, bleibt davon unber\"uhrt. Die Hochschule ist nicht verpflichtet, die Arbeit zu ver\"offentlichen. Das Einverst\"andnis kann jederzeit schriftlich (per Brief!) widerrufen werden. Die Bibliothek wird die Arbeit dann unverz\"uglich aus dem OPAC oder der Auslage entfernen. Die Ver\"offentlichung h\"angt au{\ss}erdem von der sp\"ater erteilten Zustimmung des Erstbetreuers ab.

\bigskip
\bigskip

\begin{tabular}{lp{1cm}l}
    \hspace{6cm} && \hspace{6cm} \\ \cline{1-1} \cline{3-3}
    Ort, Datum && Unterschrift
\end{tabular}

\bigskip
\bigskip

Ich bin mit der Ver\"offentlichung in dem o.g. Umfang (wird erst nach der Beurteilung ausgef\"ullt)\\
\Square \hspace{0.2cm} einverstanden. \hspace{1cm} \Square \hspace{0.2cm} n i c h t einverstanden.

\bigskip
\bigskip

\begin{tabular}{l}
    \hspace{6cm} \\ \cline{1-1}
    Erstbetreuer
\end{tabular}


    % Acknowledgement / Danksagung
    % ==================================================
    %\cleardoubleemptypage
    % ==============================================================================
%  ____              _
% |  _ \  __ _ _ __ | | _____  __ _  __ _ _   _ _ __   __ _
% | | | |/ _` | '_ \| |/ / __|/ _` |/ _` | | | | '_ \ / _` |
% | |_| | (_| | | | |   <\__ \ (_| | (_| | |_| | | | | (_| |
% |____/ \__,_|_| |_|_|\_\___/\__,_|\__, |\__,_|_| |_|\__, |
%                                   |___/             |___/
% ==============================================================================
\chapter*{Danksagung}
\label{cha:Danksagung}
% ==============================================================================

Umfang: maximal eine Seite.


    % Registers / Verzeichnisse
    % ==================================================
    %\cleardoubleemptypage
    \begin{spacing}{1.0}
        \tableofcontents                % Inhaltsverzeichnis.
    \end{spacing}
    %\cleardoubleemptypage
    % ==============================================================================
%  ____                  _           _                        _      _           _
% / ___| _   _ _ __ ___ | |__   ___ | |_   _____ _ __ _______(_) ___| |__  _ __ (_)___
% \___ \| | | | '_ ` _ \| '_ \ / _ \| \ \ / / _ \ '__|_  / _ \ |/ __| '_ \| '_ \| / __|
%  ___) | |_| | | | | | | |_) | (_) | |\ V /  __/ |   / /  __/ | (__| | | | | | | \__ \
% |____/ \__, |_| |_| |_|_.__/ \___/|_| \_/ \___|_|  /___\___|_|\___|_| |_|_| |_|_|___/
%        |___/
% ==============================================================================
\chapter*{Symbolverzeichnis}
\addcontentsline{toc}{chapter}{Symbolverzeichnis}
\label{cha:symbolverzeichnis}
% ==============================================================================

\begin{itemize}
  \item In der Arbeit verwendete Symbole and Abk"urzungen.
  \item Wenn m\"oglich Internationalen SI-Einheiten verwenden.
  \item Jeweilige Dimension, falls vorhanden, hinzuzuf\"ugen.
\end{itemize}

\bigskip
\bigskip

\begin{tabbing}
  \hspace*{5.0cm} \= \hspace*{8.0cm} \= \hspace*{3.0cm} \kill
  \textit{Abk\"urzung} \> \textit{Variable}\> \textit{Einheit} \\
  $$ \> \> \\

  $m$     \> Masse                           \> \\
  $a$     \> Probendicke                     \> mm \\
  $b$     \> Probenbreite                    \> mm \\
  $l_{A}$ \> Spannl\"ange                    \> mm \\
  $l_{B}$ \> L\"ange des Bezugsbalkens       \> mm \\
  $D_{L}$ \> Ab. zw. Bezugsbalken und Balken \> mm \\
  $X_{H}$ \> Ende der Biegemodulermittlung   \> kN \\
  $X_{L}$ \> Beginn der Biegemodulermittlung \> kN \\

\end{tabbing}

    \begin{spacing}{1.0}
        %\cleardoubleemptypage
        \listoffigures                  % Abbildungsverzeichnis.
        %\cleardoubleemptypage
        \listoftables                   % Tabellenverzeichnis.
    \end{spacing}

    % Abstract / Kurzfassung
    % ==================================================
    %\cleardoubleemptypage
    % ==============================================================================
%  _  __               __
% | |/ /   _ _ __ ____/ _| __ _ ___ ___ _   _ _ __   __ _
% | ' / | | | '__|_  / |_ / _` / __/ __| | | | '_ \ / _` |
% | . \ |_| | |   / /|  _| (_| \__ \__ \ |_| | | | | (_| |
% |_|\_\__,_|_|  /___|_|  \__,_|___/___/\__,_|_| |_|\__, |
%                                                   |___/
% ==============================================================================
\chapter*{Kurzfassung}
\addcontentsline{toc}{chapter}{Kurzfassung}
\label{cha:kurzfassung}
% ==============================================================================

Eine Kurzfassung der Arbeit.

Umfang 1 Seite.

    %\cleardoubleemptypage
    % ==============================================================================
%     _    _         _                  _
%    / \  | |__  ___| |_ _ __ __ _  ___| |_
%   / _ \ | '_ \/ __| __| '__/ _` |/ __| __|
%  / ___ \| |_) \__ \ |_| | | (_| | (__| |_
% /_/   \_\_.__/|___/\__|_|  \__,_|\___|\__|
% ==============================================================================
\chapter*{Abstract}
\addcontentsline{toc}{chapter}{Abstract}
\label{cha:abstract}
% ==============================================================================

A short version of the work in english.

Length 1 page.
 % Kurzfassung in Englisch.

    % Main chapters / Hauptkapitel
    % ==================================================
    %\cleardoubleplainpage
    \mainmatter                  % Hauptinhalt des Dokuments. Seitennummerierung in arabische Ziffern.
    % ==============================================================================
%  _____ _       _      _ _
% | ____(_)_ __ | | ___(_) |_ _   _ _ __   __ _
% |  _| | | '_ \| |/ _ \ | __| | | | '_ \ / _` |
% | |___| | | | | |  __/ | |_| |_| | | | | (_| |
% |_____|_|_| |_|_|\___|_|\__|\__,_|_| |_|\__, |
%                                         |___/
% ==============================================================================
\chapter{Einleitung}
\label{cha:einleitung}
% ==============================================================================

Folgender Inhalt sollte die Einleitung umfassen:

\begin{itemize}
  \item Kurze vorstellung der Problemstellung.
  \item Eventuell Bez\"uge zu anderen Arbeiten.
  \item Wissenschaftliche Hintergrund.
  \item Stand der Technik.
  \item Keine Methoden erl\"autern.
\end{itemize}

L\"ange maximal eine Seite.

    % ==============================================================================
%   ____                      _ _
%  / ___|_ __ _   _ _ __   __| | | __ _  __ _  ___ _ __
% | |  _| '__| | | | '_ \ / _` | |/ _` |/ _` |/ _ \ '_ \
% | |_| | |  | |_| | | | | (_| | | (_| | (_| |  __/ | | |
%  \____|_|   \__,_|_| |_|\__,_|_|\__,_|\__, |\___|_| |_|
%                                       |___/
% ==============================================================================
\chapter{Grundlagen}
\label{cha:grundlagen}
% ==============================================================================

In dem folgenden Kapitel wird sich mit allen Grundlagen auseinander gesetzt, die f\"ur das Verst\"andnis, sowie die Durchf\"uhrung der Bachelorarbeit ben\"otigt werden.

% ==============================================================================
% Unterkapitel
% ==============================================================================
\section{Unterkapitel}
% ==============================================================================

Kurze Erl\"auterung des Unterkapitels.

% ==============================================================================
% Unterunterkapitel
% ==============================================================================
\subsection{Unterunterkapitel}
% ==============================================================================

Der Gro{\ss}teil des Textes soll hier stehen.

% ==============================================================================
% UnterUnterUnterkapitel
% ==============================================================================
\subsubsection{UnterUnterUnterkapitel}
% ==============================================================================

\begin{itemize}
  \item Auf diese Ebene (UnterUnterUnterkapitel) sollte nach M\"oglichkeit verzichtet werden.
  \item Eine Untergliederung darf nur erfolgen, wenn mindestens zwei Punkte auf einer Ebene vorliegen.
\end{itemize}

Das ist hier ist eine test Referenz \cite{SalmangScholze}.

    % ==============================================================================
%  _   _                   _   _       _ _
% | | | | __ _ _   _ _ __ | |_| |_ ___(_) |
% | |_| |/ _` | | | | '_ \| __| __/ _ \ | |
% |  _  | (_| | |_| | |_) | |_| ||  __/ | |
% |_| |_|\__,_|\__,_| .__/ \__|\__\___|_|_|
%                   |_|
% ==============================================================================
\chapter{Hauptteil}
\label{cha:Hauptteil}
% ==============================================================================

In diesem Abschnitt werden alle Daten und Ergebnisse, die im Verlauf der Arbeit gesammelt wurden, eingef\"ugt und beschrieben.

    % ==============================================================================
%  ____       _     _
% / ___|  ___| |__ | |_   _ ___ ___
% \___ \ / __| '_ \| | | | / __/ __|
%  ___) | (__| | | | | |_| \__ \__ \
% |____/ \___|_| |_|_|\__,_|___/___/
% ==============================================================================
\chapter{Schluss}
\label{cha:Schluss}
% ==============================================================================

Eine kurze Zusammenfassung der wichtigsten Ergebnisse.
Dieser Abschnitt soll auch verwendet werden um offene Fragen und Probleme betreffend der Arbeit zu erl\"autern.

Umfang: 1-3 Seiten.


    % Bibliography / Literaturverzeichnis
    % ==================================================
    %\cleardoubleplainpage
    \printbibliography[title=Literaturverzeichnis]

    % Appendix / Anhang
    % ==================================================
    %\cleardoubleplainpage
    % ==============================================================================
%     _          _
%    / \   _ __ | |__   __ _ _ __   __ _
%   / _ \ | '_ \| '_ \ / _` | '_ \ / _` |
%  / ___ \| | | | | | | (_| | | | | (_| |
% /_/   \_\_| |_|_| |_|\__,_|_| |_|\__, |
%                                  |___/
% ==============================================================================
\chapter*{Anhang}
\addcontentsline{toc}{chapter}{Anhang}
\label{cha:anhang}
% ==============================================================================

In diesem Abschnitt soll der Arbeit z.B. folgendes hinzugef\"ugt werden:

\begin{itemize}
  \item Bedienungsanleitungen
  \item Konstruktionszeichnungen
  \item Schaltpl\"ane
\end{itemize}


    % Blank page / Leerseite
    % ==================================================
    \newpage
    \thispagestyle{empty}
    \quad
    \newpage

\end{document}
