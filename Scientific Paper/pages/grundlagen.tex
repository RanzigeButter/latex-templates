% ==============================================================================
%   ____                      _ _
%  / ___|_ __ _   _ _ __   __| | | __ _  __ _  ___ _ __
% | |  _| '__| | | | '_ \ / _` | |/ _` |/ _` |/ _ \ '_ \
% | |_| | |  | |_| | | | | (_| | | (_| | (_| |  __/ | | |
%  \____|_|   \__,_|_| |_|\__,_|_|\__,_|\__, |\___|_| |_|
%                                       |___/
% ==============================================================================
\chapter{Grundlagen}
\label{cha:grundlagen}
% ==============================================================================

In dem folgenden Kapitel wird sich mit allen Grundlagen auseinander gesetzt, die f\"ur das Verst\"andnis, sowie die Durchf\"uhrung der Bachelorarbeit ben\"otigt werden.

% ==============================================================================
% Unterkapitel
% ==============================================================================
\section{Unterkapitel}
% ==============================================================================

Kurze Erl\"auterung des Unterkapitels.

% ==============================================================================
% Unterunterkapitel
% ==============================================================================
\subsection{Unterunterkapitel}
% ==============================================================================

Der Gro{\ss}teil des Textes soll hier stehen.

% ==============================================================================
% UnterUnterUnterkapitel
% ==============================================================================
\subsubsection{UnterUnterUnterkapitel}
% ==============================================================================

\begin{itemize}
  \item Auf diese Ebene (UnterUnterUnterkapitel) sollte nach M\"oglichkeit verzichtet werden.
  \item Eine Untergliederung darf nur erfolgen, wenn mindestens zwei Punkte auf einer Ebene vorliegen.
\end{itemize}

Das ist hier ist eine test Referenz \cite{SalmangScholze}.
