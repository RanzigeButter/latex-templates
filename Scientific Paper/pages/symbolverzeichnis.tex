% ==============================================================================
%  ____                  _           _                        _      _           _
% / ___| _   _ _ __ ___ | |__   ___ | |_   _____ _ __ _______(_) ___| |__  _ __ (_)___
% \___ \| | | | '_ ` _ \| '_ \ / _ \| \ \ / / _ \ '__|_  / _ \ |/ __| '_ \| '_ \| / __|
%  ___) | |_| | | | | | | |_) | (_) | |\ V /  __/ |   / /  __/ | (__| | | | | | | \__ \
% |____/ \__, |_| |_| |_|_.__/ \___/|_| \_/ \___|_|  /___\___|_|\___|_| |_|_| |_|_|___/
%        |___/
% ==============================================================================
\chapter*{Symbolverzeichnis}
\addcontentsline{toc}{chapter}{Symbolverzeichnis}
\label{cha:symbolverzeichnis}
% ==============================================================================

\begin{itemize}
  \item In der Arbeit verwendete Symbole and Abk"urzungen.
  \item Wenn m\"oglich Internationalen SI-Einheiten verwenden.
  \item Jeweilige Dimension, falls vorhanden, hinzuzuf\"ugen.
\end{itemize}

\bigskip
\bigskip

\begin{tabbing}
  \hspace*{5.0cm} \= \hspace*{8.0cm} \= \hspace*{3.0cm} \kill
  \textit{Abk\"urzung} \> \textit{Variable}\> \textit{Einheit} \\
  $$ \> \> \\

  $m$     \> Masse                           \> \\
  $a$     \> Probendicke                     \> mm \\
  $b$     \> Probenbreite                    \> mm \\
  $l_{A}$ \> Spannl\"ange                    \> mm \\
  $l_{B}$ \> L\"ange des Bezugsbalkens       \> mm \\
  $D_{L}$ \> Ab. zw. Bezugsbalken und Balken \> mm \\
  $X_{H}$ \> Ende der Biegemodulermittlung   \> kN \\
  $X_{L}$ \> Beginn der Biegemodulermittlung \> kN \\

\end{tabbing}
